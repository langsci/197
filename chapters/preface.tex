\addchap{\lsPrefaceTitle}
  

How should we think about time, for the purpose of thinking about language? Do we even need to think about time? This book says “yes, we need to think about time, and we need to think about time differently.” It is not the case that there is a “right way” of thinking about time, but there are less familiar, alternative ways of thinking about time. This is important because the ways we think about time, and the ways we think about language, interact. It is not really possible to think about language, in any deep sense, without thinking about time, whether we are aware of it or not. 

  Thinking about time and language is not separable from the vocabulary we use. The metaphors we use in a vocabulary arise from a conceptual framework which always pre-determines our analyses of phenomena. To illustrate, consider the following passages from Bar-\citet{Hillel1953a}:

“Each sentence that is not an element is regarded as the outcome of the operation of one sub-sequence upon the remainder, which may be to its immediate right or to its immediate left or on both sides. ('Left' and 'right' are to be understood here, as in what follows, only as the two directions of a linear order.)” (1953a: 50).

“If we write \textit{Paul, strangely enough, refused to talk} (which is, incidentally, the common usage), and interpret the function of the commas as giving us license to lift the string between them from its position and deposit it at some other position (within certain limits, of course), we can still adhere to the simple rules of immediate environment. It remains to be seen whether devices of such a simple nature will enable us to retain a notation which takes account only of the immediate environment with respect to all languages” (1953a: 58).

There is a lot that we might unpack from these passages, but I would like highlight two metaphors: \textsc{time} is \textsc{space} and \textsc{words} are \textsc{objects}, along with a blend of these metaphors: \textsc{temporal order of words} is \textsc{spatial arrangement of objects}. These metaphors are evident in the use of phrases like “immediate right” and the “position” of a string. There are generic images which are evoked by these metaphors, and my contention is that these images predetermine the ways in which the author can reason about language. 

The metaphors in the above passage are ancient and possibly co-originate with the technology of writing; they certainly predate the origins of modern linguistics. Yet these same metaphors are dominant to this day. Despite their longevity and popularity, I believe it is our responsibility to call them into question, and to pursue alternatives. The most direct way of doing this is to analyze the vocabularies of current theories, and to attempt to develop an alternative vocabulary that evokes different images. I suspect that some readers will be uncomfortable with the emphasis placed on vocabularies and metaphors in this book. This is understandable because it is not conventional in our current discourse to discuss such things. Indeed, a shared vocabulary and system of metaphors is required for normal science, and questioning that vocabulary undermines the enterprise. Thus, this book is not “normal science” in the sense that it does not presuppose the prevailing metaphors of the day.

